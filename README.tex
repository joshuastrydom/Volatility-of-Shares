% Options for packages loaded elsewhere
\PassOptionsToPackage{unicode}{hyperref}
\PassOptionsToPackage{hyphens}{url}
%
\documentclass[
]{article}
\usepackage{amsmath,amssymb}
\usepackage{lmodern}
\usepackage{iftex}
\ifPDFTeX
  \usepackage[T1]{fontenc}
  \usepackage[utf8]{inputenc}
  \usepackage{textcomp} % provide euro and other symbols
\else % if luatex or xetex
  \usepackage{unicode-math}
  \defaultfontfeatures{Scale=MatchLowercase}
  \defaultfontfeatures[\rmfamily]{Ligatures=TeX,Scale=1}
\fi
% Use upquote if available, for straight quotes in verbatim environments
\IfFileExists{upquote.sty}{\usepackage{upquote}}{}
\IfFileExists{microtype.sty}{% use microtype if available
  \usepackage[]{microtype}
  \UseMicrotypeSet[protrusion]{basicmath} % disable protrusion for tt fonts
}{}
\makeatletter
\@ifundefined{KOMAClassName}{% if non-KOMA class
  \IfFileExists{parskip.sty}{%
    \usepackage{parskip}
  }{% else
    \setlength{\parindent}{0pt}
    \setlength{\parskip}{6pt plus 2pt minus 1pt}}
}{% if KOMA class
  \KOMAoptions{parskip=half}}
\makeatother
\usepackage{xcolor}
\IfFileExists{xurl.sty}{\usepackage{xurl}}{} % add URL line breaks if available
\IfFileExists{bookmark.sty}{\usepackage{bookmark}}{\usepackage{hyperref}}
\hypersetup{
  hidelinks,
  pdfcreator={LaTeX via pandoc}}
\urlstyle{same} % disable monospaced font for URLs
\usepackage[margin=1in]{geometry}
\usepackage{graphicx}
\makeatletter
\def\maxwidth{\ifdim\Gin@nat@width>\linewidth\linewidth\else\Gin@nat@width\fi}
\def\maxheight{\ifdim\Gin@nat@height>\textheight\textheight\else\Gin@nat@height\fi}
\makeatother
% Scale images if necessary, so that they will not overflow the page
% margins by default, and it is still possible to overwrite the defaults
% using explicit options in \includegraphics[width, height, ...]{}
\setkeys{Gin}{width=\maxwidth,height=\maxheight,keepaspectratio}
% Set default figure placement to htbp
\makeatletter
\def\fps@figure{htbp}
\makeatother
\setlength{\emergencystretch}{3em} % prevent overfull lines
\providecommand{\tightlist}{%
  \setlength{\itemsep}{0pt}\setlength{\parskip}{0pt}}
\setcounter{secnumdepth}{-\maxdimen} % remove section numbering
\ifLuaTeX
  \usepackage{selnolig}  % disable illegal ligatures
\fi

\author{}
\date{\vspace{-2.5em}}

\begin{document}

\hypertarget{volatility-of-shares}{%
\section{Volatility-of-Shares}\label{volatility-of-shares}}

This README is here to describe my analysis with the accompanying code.
The aim of the paper is to compare of the volatility of returns for
large, mid and small cap shares in South Africa. The analysis will be
conducted using many packages, with particular use of the
PerformanceAnalytics package. Indexes for Large Cap, Mid Cap and Small
Cap shares will be created by calculating the product of each
constituent shares return and its weight in the respective index.
Return, standard deviation and correlation calculations will be used to
compare the indexes.

\hypertarget{packages}{%
\section{Packages}\label{packages}}

\texttt{\{r\ setup,\ include=FALSE\}\ if(!require("tidyverse"))\ install.packages("tidyverse")\ library(tidyverse)\ if(!require("tidyverse"))\ install.packages("tidyverse")\ library(tidyverse)\ library(quantmod)\ library(ggplot2)\ library(dplyr)\ library(tidyverse)\ library(fmxdat)\ library(data.table)\ library(tbl2xts)\ library(tidyr)\ library(PerformanceAnalytics)\ library(rvest)\ library(corrplot)\ library(ggpubr)}
\# Load the data
\texttt{\{r\ Load\ data\}\ Alsi\ \textless{}-\ readRDS("data/Alsi\_Returns.rds")\ Alsi\ \textless{}-\ as.data.frame(Alsi)}
\# Create functions

```\{r Functions\} \#Create functions that create the data frames for
each size contribution \textless- function(x) \{ index1 \textless-
x{[},c(1:3,7){]} \textbar\textgreater{} mutate(Contribution = Return *
J203) \#All share index index1{[}is.na(index1){]} = 0 wide\_index1
\textless- spread(index1{[},c(1:2,5){]}, Tickers, Contribution)
wide\_index1{[}is.na(wide\_index1){]} = 0 daily\_contribution \textless-
wide\_index1 \textbar\textgreater{} mutate(Daily\_return =
select(wide\_index1, 2:ncol(wide\_index1)) \textbar\textgreater{}
rowSums(na.rm = TRUE)) \#Sum of daily returns of each stock for index
first \textless- daily\_contribution{[},1{]} last \textless-
daily\_contribution{[},ncol(daily\_contribution){]} Total \textless-
data.frame(first,last) colnames(Total){[}1{]} \textless- ``date''
colnames(Total){[}2{]} \textless- ``Daily\_return'' return(Total) \}
\#Use output for rolling and daily functions

contributionmid \textless- function(x) \{ index1 \textless-
x{[},c(1:3,9){]} \textbar\textgreater{} mutate(Contribution = Return *
J201) \#Mid cap index index1{[}is.na(index1){]} = 0 wide\_index1
\textless- spread(index1{[},c(1:2,5){]}, Tickers, Contribution)
wide\_index1{[}is.na(wide\_index1){]} = 0 daily\_contribution \textless-
wide\_index1 \textbar\textgreater{} mutate(J201Daily\_return =
select(wide\_index1, 2:ncol(wide\_index1)) \textbar\textgreater{}
rowSums(na.rm = TRUE)) \#Sum of daily returns of each stock for index
Total \textless- daily\_contribution{[},c(1,177){]} return(Total) \}
\#Use output for rolling and daily functions

contributionsmall \textless- function(x) \{ index1 \textless-
x{[},c(1:3,10){]} \textbar\textgreater{} mutate(Contribution = Return *
J202) \#Small cap index index1{[}is.na(index1){]} = 0 wide\_index1
\textless- spread(index1{[},c(1:2,5){]}, Tickers, Contribution)
wide\_index1{[}is.na(wide\_index1){]} = 0 daily\_contribution \textless-
wide\_index1 \textbar\textgreater{} mutate(J202Daily\_return =
select(wide\_index1, 2:ncol(wide\_index1)) \textbar\textgreater{}
rowSums(na.rm = TRUE)) \#Sum of daily returns of each stock for index
Total \textless- daily\_contribution{[},c(1,225){]} return(Total) \}
\#Use output for rolling and daily functions

daily\_contri \textless- function(x)\{ daily \textless- x
\textbar\textgreater{} mutate(Cum\_Ret = cumprod(1 + Daily\_return)) xts
\textless- tbl\_xts(daily\_contri\_large1) return(xts) \}

rolling\_contri \textless- function(x)\{ rolling \textless- x
\textbar\textgreater{} gather(Ticker, Contribution, -date)
\textbar\textgreater{} group\_by(Ticker) xts \textless- tbl\_xts(tblData
= rolling, cols\_to\_xts = Contribution, spread\_by = Ticker)
return(xts) \}

rolling\_SDcum \textless- function(x)\{ rolling \textless- x
\textbar\textgreater{} \#Use large\_caps, mid\_caps or small\_caps
mutate(YM = format(date, ``\%Y\%B'')) \textbar\textgreater{}
arrange(date) \textbar\textgreater{} group\_by(Tickers, YM)
\textbar\textgreater{} filter(date == last(date)) \textbar\textgreater{}
group\_by(Tickers) \textbar\textgreater{} select(date, Tickers, Return)
\textbar\textgreater{} mutate(RollSD = RcppRoll::roll\_sd(1 + Return,
36, fill = NA, align = ``right'') * sqrt(12)) \textbar\textgreater{}
filter(!is.na(RollSD)) return(rolling) \}

\begin{verbatim}
# Dataframe creation 

The following code chunk was used to find the daily returns for each index given the constituent shares' respective weight in each appropriate index. Large Cap shares are part of the J200, Mid Cap shares are part of the J201 and Small Cap shares are part of the J202. 

```{r Dataframes}
# Wrote function to find daily returns for each index given the constituent shares. 
index <- function(x){
    index <- x
    index <- index[,-c(5:10)]
    index$Tickers <- gsub(" SJ|Equity","", index$Tickers)
    wide <- spread(index[,c(1,2,7)], Tickers, Contribution)
    Total <- wide |> 
      mutate(Contribution = select(wide, 2:ncol(wide)) |> rowSums(na.rm = TRUE)) #Sum of daily returns of all stocks of the index
    first <- Total[,1] #Kept the date column
    last <- Total[,ncol(Total)] #Kept the contribution column which ended last due to the mutate
    Final <- data.frame(first,last) #Merged the dateframes
    colnames(Final)[1] <- "date"
    colnames(Final)[2] <- "Daily_return"
    Cum <- Final |> 
        mutate(Cumulative_returns = cumprod(1 + Daily_return)) |>#Get cumulative returns for the index
        filter(Daily_return != 0) #Filtered out rows that did not contribute. 
    return(Cum)
}
J200 <- Alsi |> 
    filter(Index_Name == "Large_Caps") |> 
    mutate(Contribution = Return*J200)
l_caps <- index(J200) #Large Cap Index

J201 <- Alsi |> 
    filter(Index_Name == "Mid_Caps") |> 
    mutate(Contribution = Return*J201)
m_caps <- index(J201) #Mid Cap Index
J202 <- Alsi |> 
    filter(Index_Name == "Small_Caps") |> 
    mutate(Contribution = Return*J202)
s_caps <- index(J202) #Small Cap Index
\end{verbatim}

\hypertarget{index-analysis}{%
\section{Index Analysis}\label{index-analysis}}

The analysis of volatility is based on three indexes of the Johannesburg
Securities Exchange (JSE). Data is attained by filtering out the FTSE
All Share Index (ALSI). The large cap index is based on the FTSE/JSE Top
40 Index (J200), the mid cap index is based on the FTSE/JSE Mid Cap
Index (J201) and the small cap index is based on the FTSE/JSE Small Cap
Index (J202). The derived returns for each of the separate indexes are
calculated as the product of a stocks daily return and the assigned
weight in the respective index. For the large and small cap indexes,
data from 2005 to 2022 is available. Unfortunately for the mid cap
index, data is only available from 2016 onward.

\hypertarget{drawdown-calculations}{%
\subsection{Drawdown calculations}\label{drawdown-calculations}}

\texttt{\{r\ Drawdowns\}\ \#Had\ to\ convert\ the\ previously\ composed\ indexes\ into\ an\ xts\ format\ daily\_contri\ \textless{}-\ function(x)\{\ \ \ \ \ xts\ \textless{}-\ tbl\_xts(x)\ \ \ \ \ return(xts)\ \}\ \#Created\ spearte\ dataframes\ for\ each\ index\ drawdown\_large\ \textless{}-\ daily\_contri(l\_caps)\ drawdown\_mid\ \textless{}-\ daily\_contri(m\_caps)\ drawdown\_small\ \textless{}-\ daily\_contri(s\_caps)}
Below is the Large Cap Index drawdown graph.
\texttt{\{r\ Figure1\}\ chart.Drawdown(drawdown\_large\$Daily\_return,main\ =\ "Drawdowns:\ Large\ Caps",\ \ \ \ \ \ \ \ \ \ \ \ \ \ \ \ \ col\ =\ "steelblue")}
Below is the Mid Cap Index drawdown graph.
\texttt{\{r\ Figure2\}\ chart.Drawdown(drawdown\_mid\$Daily\_return,main\ =\ "Drawdowns:\ Mid\ Caps",\ \ \ \ \ \ \ \ \ \ \ \ \ \ \ \ \ col\ =\ "steelblue")}
Below is the Small Cap Index drawdown graph.
\texttt{\{r\ Figure3\}\ chart.Drawdown(drawdown\_small\$Daily\_return,main\ =\ "Drawdowns:\ Small\ Caps",\ \ \ \ \ \ \ \ \ \ \ \ \ \ \ \ \ col\ =\ "steelblue")}
\#\# Sector weightings

Sector weightings calculations were used to identify and anaylse the
impct of index composition on the return profile, standard deviation and
degree of correlation in the index.

\texttt{\{r\ Weightings,\ include=FALSE\}\ \#Calcualtions\ for\ the\ weight\ comparision\ chart\ for\ sectors\ of\ each\ index.\ New\ dataframes\ were\ created\ so\ that\ the\ accompanying\ function\ could\ operate\ effectively.\ \ J200sectors\ \textless{}-\ J200{[},c(1,11,8){]}\ \textbar{}\textgreater{}\ \ \ \ \ \ group\_by(Sector)\ \textbar{}\textgreater{}\ \ \ \ \ \ rename(Weight\ =\ J200)\ J201sectors\ \textless{}-\ J201{[},c(1,11,9){]}\ \textbar{}\textgreater{}\ \ \ \ \ \ group\_by(Sector)\ \textbar{}\textgreater{}\ \ \ \ \ \ rename(Weight\ =\ J201)\ J202sectors\ \textless{}-\ J202{[},c(1,11,10){]}\ \textbar{}\textgreater{}\ \ \ \ \ \ group\_by(Sector)\ \textbar{}\textgreater{}\ \ \ \ \ \ rename(Weight\ =\ J202)\ \#Created\ a\ function\ to\ return\ the\ data\ used\ to\ construct\ the\ sector\ weightings\ within\ each\ index.\ \ \ \ \ \ sector\ \textless{}-\ function(x)\{\ \ \ \ \ names\ =\ c("Financials","Industrials","Resources","Property")\ \ \ \ \ filtered\ \textless{}-\ x\ \ \ \ \ weights\ \textless{}-\ aggregate(Weight\textasciitilde{}.,\ x,FUN=sum)\ \textbar{}\textgreater{}\ \ \ \ \ \ \ \ \ \ group\_by(date)\ \textbar{}\textgreater{}\ \ \ \ \ \ \ \ \ \ ggplot()\ +\ \ \ \ \ \ \ \ \ geom\_bar(aes(x=date,\ y=Weight,\ fill=factor(Sector,\ names)),\ \ \ \ \ \ \ \ \ \ \ \ \ \ \ \ \ \ stat\ =\ \textquotesingle{}identity\textquotesingle{},\ width\ =\ 1)\ +\ \ \ \ \ \ \ \ \ scale\_y\_continuous(labels\ =\ scales::percent\_format(scale\ =\ 100))\ +\ \ \ \ \ \ \ \ \ labs(x\ =\ NULL,\ y\ =\ NULL,\ title\ =\ "Sector\ Weighting",\ fill\ =\ "Sector")\ +\ \ \ \ \ \ \ \ \ theme\_bw()\ +\ \ \ \ \ \ \ \ \ theme(legend.position="bottom")\ \ \ \ \ \ \ \ \ return(weights)\ \}\ J200plot\ \textless{}-\ sector(J200sectors)\ \#Large\ Cap\ Index\ J201plot\ \textless{}-\ sector(J201sectors)\ \#Mid\ Cap\ Index\ J202plot\ \textless{}-\ sector(J202sectors)\ \#Small\ Cap\ Index}
The following code chunk was used to determine the mean return for each
sector irrespective of their weight in any index.
\texttt{\{r\ sectors,\ include=FALSE\}\ sectors\ \textless{}-\ Alsi{[},c(1,3,11){]}\ \textbar{}\textgreater{}\ \ \ \ \ \ group\_by(date,\ Sector)\ \textbar{}\textgreater{}\ \ \ \ \ \ summarise(Sum\ =\ sum(Return))\ \textbar{}\textgreater{}\ \ \ \ \ \ ungroup()\ \textbar{}\textgreater{}\ \ \ \ \ \ filter(!is.na(Sector))\ \textbar{}\textgreater{}\ \ \ \ \ \ na.omit()\ \textbar{}\textgreater{}\ \ \ \ \ \ spread(Sector,\ Sum)\ summary(sectors)}
The results of the mean return calcualtion were manually tabulated as
follows:

\begin{table}[h]
\begin{center}
    \begin{tabular}{| c | c |}
    \hline
        Sector & Mean \\
        \hline
        Financials & 0.16530 \\
        Industrials & 1.342 \\
        Property & 0.009539 \\
        Resources & 0.07961 \\
        \hline
    \end{tabular}
    \caption{Mean daily returns}
    \label{tab:Mean}
\end{center}
\end{table}

Now, the weighting plots are displayed in order to analyse each separate
index in detail. \texttt{\{r\ Figure4\}\ J200plot}
\texttt{\{r\ Figure5\}\ J201plot} \texttt{\{r\ Figure6\}\ J202plot} \#
Annualised returns

At first, the rolling annualised returns of the indexes were calculated
using various functions. The resultant dataframes produced by the
functions were used to plot the rolling annualised return graph.

```\{r Annual returns\} \#Rolling annualised returns of indices
roll\_annualreturn \textless- function(x,y,z)\{ large \textless-
x{[},-c(3){]} colnames(large){[}2{]} \textless- ``Large\_Cap'' mid
\textless- y{[},-c(3){]} colnames(mid){[}2{]} \textless- ``Mid\_Cap''
small \textless- z{[},-c(3){]} colnames(small){[}2{]} \textless-
``Small\_Cap'' joined \textless- list(large,mid,small)
\textbar\textgreater{} reduce(full\_join, by=`date')
\textbar\textgreater{} gather(Index, Return, -date)
\textbar\textgreater{} arrange(date) \textbar\textgreater{}
filter(!is.na(Return)) \textbar\textgreater{} mutate(YM = format(date,
``\%Y\%B'')) \textbar\textgreater{} arrange(date) \textbar\textgreater{}
group\_by(Index, YM) \textbar\textgreater{} filter(date == last(date))
return(joined) \} \#Joined the separate datasets crated. full\_return
\textless- roll\_annualreturn(l\_caps, m\_caps, s\_caps) plotreturn
\textless- full\_return \textbar\textgreater{} group\_by(Index)
\textbar\textgreater{} mutate(RollRets = RcppRoll::roll\_prod(1 +
Return, 36, fill = NA, align = ``right'')\^{}(12/36) - 1)
\textbar\textgreater{} group\_by(date) \textbar\textgreater{}
filter(any(!is.na(RollRets))) \textbar\textgreater{} ungroup()
\textbar\textgreater{} ggplot() + geom\_line(aes(date, RollRets, color =
Index), alpha = 0.7, size = 1.25) + labs(title = ``Rolling 3 Year
Annualized Returns with differing start dates'', subtitle = ``Large, Mid
and Small cap Indexes'', x = ````, y =''Rolling 3 year Returns (Ann.)``,
caption =''Note:\nDistortions are not evident now.'') +
theme\_fmx(title.size = ggpts(30), subtitle.size = ggpts(20),
caption.size = ggpts(25), CustomCaption = T) + fmx\_cols()

\begin{verbatim}
I then plotted the joined dataset of the annualised returns for the three separate indexes. 

```{r Figure7}
finplot(plotreturn, x.date.dist = "1 year", x.date.type = "%Y", x.vert = T, y.pct = T, y.pct_acc = 1)
\end{verbatim}

A density plot of the distribution of annualised returns seemed
necessary. The code used to create such a plot follows below. ```\{r ret
compare\} b \textless- full\_return \textbar\textgreater{}
group\_by(Index) \textbar\textgreater{} mutate(RollRets =
RcppRoll::roll\_prod(1 + Return, 12, fill = NA, align =
``right'')\^{}(12/12) - 1) \textbar\textgreater{} \#Annualised rolling
returns group\_by(date) \textbar\textgreater{}
filter(any(!is.na(RollRets))) \textbar\textgreater{} ungroup()

\hypertarget{c-is-used-as-the-input-for-each-of-the-density-plots.}{%
\section{c is used as the input for each of the density
plots.}\label{c-is-used-as-the-input-for-each-of-the-density-plots.}}

c \textless- b{[}-c(1:146),-c(3,4){]} \textbar\textgreater{} \#Filtered
out for the past 10 years and got rid of unnecessary columns
spread(Index, RollRets) return\_density \textless- full\_return{[},-4{]}
\textbar\textgreater{} spread(Index, Return) \#Large Cap Index density
plot return\_density\_large \textless- ggplot(c{[},c(1,2){]}, aes(x =
Large\_Cap)) + geom\_density(color = ``darkred'', fill = ``red'') +
geom\_vline(aes(xintercept = mean(Large\_Cap)), linetype = ``dashed'',
size = 0.6) + coord\_flip() + scale\_x\_continuous(limits = c(-0.05,
0.1), labels = scales::percent\_format(scale = 100)) +
theme(axis.text.x=element\_blank(), axis.ticks.x=element\_blank(),
axis.title.x = element\_blank(), axis.title.y = element\_blank(),
panel.grid.major = element\_blank(), panel.grid.minor =
element\_blank(), panel.background = element\_blank()) +
theme(legend.position=``none'')

\#Mid Cap Index density plot mid\_density \textless- c{[},c(1,3){]}
\textbar\textgreater{} na.omit() \#Had to omit NAs for the geom\_vline
to function. return\_density\_mid \textless- ggplot(mid\_density, aes(x
= Mid\_Cap)) + geom\_density(color = ``darkblue'', fill = ``lightblue'')
+ geom\_vline(aes(xintercept = mean(Mid\_Cap)), linetype = ``dashed'',
size = 0.6, na.rm = FALSE) + coord\_flip() + scale\_x\_continuous(limits
= c(-0.05, 0.1), labels = scales::percent\_format(scale = 100)) +
theme(axis.text.x=element\_blank(), axis.ticks.x=element\_blank(),
axis.title.x = element\_blank(), axis.title.y = element\_blank(),
panel.grid.major = element\_blank(), panel.grid.minor =
element\_blank(), panel.background = element\_blank()) +
theme(legend.position=``none'')

\#Small Cap Index density plot return\_density\_small \textless-
ggplot(c{[},c(1,4){]}, aes(x = Small\_Cap)) + geom\_density(color =
``darkgreen'', fill = ``lightgreen'') + geom\_vline(aes(xintercept =
mean(Small\_Cap)), linetype = ``dashed'', size = 0.6) + coord\_flip() +
scale\_x\_continuous(limits = c(-0.05, 0.1), labels =
scales::percent\_format(scale = 100)) +
theme(axis.text.x=element\_blank(), axis.ticks.x=element\_blank(),
axis.title.x = element\_blank(), axis.title.y = element\_blank(),
panel.grid.major = element\_blank(), panel.grid.minor =
element\_blank(), panel.background = element\_blank()) +
theme(legend.position=``none'')

\begin{verbatim}
I used ggarange to combine the density plots. 
```{r Figure8}
retplot <- ggarrange(return_density_large, return_density_mid, return_density_small + rremove("x.text"), 
          labels = c("Large Cap", "Mid Cap", "Small Cap"), align = "h", common.legend = TRUE, font.label = list(size = 10, color = "black"),
          ncol = 3, nrow = 1)
annotate_figure(retplot, top = text_grob("Rolling 12-Month Return Comparison (past 10 Years)", 
               color = "black", face = "bold", size = 11))
\end{verbatim}

\hypertarget{rolling-standard-deviation}{%
\section{Rolling Standard deviation}\label{rolling-standard-deviation}}

The rolling standard deviation calculations follow. The RcppRoll
function was employed to determine the rolling standard deviations.

```\{r Join\} \#Filter out the necessary columns of the respective
indexes ljoin \textless- l\_caps{[},c(1,2){]} colnames(ljoin){[}2{]}
\textless- ``large\_dailyreturn'' mjoin \textless- m\_caps{[},c(1,2){]}
colnames(mjoin){[}2{]} \textless- ``mid\_dailyreturn'' sjoin \textless-
s\_caps{[},c(1,2){]} colnames(sjoin){[}2{]} \textless-
``small\_dailyreturn''

\#Wrote a function to calculate the rolling stanard deviations roll\_SD
\textless- function(x,y,z)\{ large \textless- x \textbar\textgreater{}
mutate(Large\_Cap = RcppRoll::roll\_sd(1 + Daily\_return, 36, fill = NA,
align = ``right'') * sqrt(12)) \textbar\textgreater{}
filter(!is.na(Large\_Cap)) l \textless- large{[},c(1,4){]} mid
\textless- y \textbar\textgreater{} mutate(Mid\_Cap =
RcppRoll::roll\_sd(1 + Daily\_return, 36, fill = NA, align = ``right'')
* sqrt(12)) \textbar\textgreater{} filter(!is.na(Mid\_Cap)) m \textless-
mid{[},c(1,4){]} small \textless- z \textbar\textgreater{}
mutate(Small\_Cap = RcppRoll::roll\_sd(1 + Daily\_return, 36, fill = NA,
align = ``right'') * sqrt(12)) \textbar\textgreater{}
filter(!is.na(Small\_Cap)) s \textless- small{[},c(1,4){]} joined
\textless- list(l,m,s) \textbar\textgreater{} reduce(full\_join,
by=`date') \textbar\textgreater{} gather(Index, SD, -date)
\textbar\textgreater{} arrange(date) return(joined) \}

\#Used the function to create a graph of the rolling standard
deviations. full\_SD \textless- roll\_SD(l\_caps, m\_caps, s\_caps)
\textbar\textgreater{} ggplot() + geom\_line(aes(x = date, y = SD, color
= Index), alpha = 0.7, size = 1) + labs(title = ``Rolling 3 Year
Annualized SD with differing starting dates'', subtitle = ``Large, Mid
and Small cap Indexes'', x = ````, y =''Rolling 3 year Returns (Ann.)``,
caption =''Note:\nDistortions are not evident now.'') +
theme\_fmx(title.size = ggpts(30), subtitle.size = ggpts(20),
caption.size = ggpts(25), CustomCaption = T) +
scale\_y\_continuous(labels = scales::percent\_format(scale = 100)) +
theme(legend.position = ``bottom'') + fmx\_cols()

\begin{verbatim}
I then used the resulting dataframes/gg dataset to plot the results. 

```{r Figure9}
finplot(full_SD, x.date.dist = "1 year", x.date.type = "%Y", x.vert = T, 
    y.pct = T, y.pct_acc = 1)
\end{verbatim}

Now, I needed to data for the rolling 12 month standard deviation of
each index. This data would eventually be used to plot density graphs.

```\{r SD compare\} \#Rolling 12-month standard deviation for the
indexes full \textless- full\_return \textbar\textgreater{}
group\_by(Index) \textbar\textgreater{} select(date, Index, Return)
\textbar\textgreater{} mutate(RollSD = RcppRoll::roll\_sd(1 + Return,
12, fill = NA, align = ``right'') * sqrt(12)) \textbar\textgreater{}
filter(!is.na(RollSD))

\#Filter out columns that are not necessary. SD\_density \textless-
full{[},-3{]} \textbar\textgreater{} spread(Index, RollSD)

\#Create density plot for the Large Cap Index sd\_density\_large
\textless- ggplot(SD\_density{[}-c(1:73),c(1,2){]}, aes(x = Large\_Cap))
+ geom\_density(color = ``darkred'', fill = ``red'') +
geom\_vline(aes(xintercept = mean(Large\_Cap)), linetype = ``dashed'',
size = 0.6) + coord\_flip() + scale\_x\_continuous(limits = c(0.01,
0.04),labels = scales::percent\_format(scale = 100)) +
theme(axis.text.x=element\_blank(), axis.ticks.x=element\_blank(),
axis.title.x = element\_blank(), axis.title.y = element\_blank(),
panel.grid.major = element\_blank(), panel.grid.minor =
element\_blank(), panel.background = element\_blank()) +
theme(legend.position=``none'')

\#Create density plot for the Mid Cap Index sd\_mid \textless-
SD\_density{[},c(1,3){]} \textbar\textgreater{} na.omit()
sd\_density\_mid \textless- ggplot(sd\_mid, aes(x = Mid\_Cap)) +
geom\_density(color = ``darkblue'', fill = ``lightblue'') +
geom\_vline(aes(xintercept = mean(Mid\_Cap)), linetype = ``dashed'',
size = 0.6, na.rm = FALSE) + coord\_flip() + scale\_x\_continuous(limits
= c(0.01, 0.04), labels = scales::percent\_format(scale = 100)) +
theme(axis.text.x=element\_blank(), axis.ticks.x=element\_blank(),
axis.title.x = element\_blank(), axis.title.y = element\_blank(),
panel.grid.major = element\_blank(), panel.grid.minor =
element\_blank(), panel.background = element\_blank()) +
theme(legend.position=``none'')

\#Create density plot for the Small Cap Index sd\_density\_small
\textless- ggplot(SD\_density{[}-c(1:73),c(1,4){]}, aes(x = Small\_Cap))
+ geom\_density(color = ``darkgreen'', fill = ``lightgreen'') +
geom\_vline(aes(xintercept = mean(Small\_Cap)), linetype = ``dashed'',
size = 0.6) + coord\_flip() + scale\_x\_continuous(limits = c(0.01,
0.04), labels = scales::percent\_format(scale = 100)) +
theme(axis.text.x=element\_blank(), axis.ticks.x=element\_blank(),
axis.title.x = element\_blank(), axis.title.y = element\_blank(),
panel.grid.major = element\_blank(), panel.grid.minor =
element\_blank(), panel.background = element\_blank()) +
theme(legend.position=``none'')

\begin{verbatim}
I used ggarrange to combine the respective ggplots for a nice figure to be used in the paper. 

```{r Figure10}
sdplot <- ggarrange(sd_density_large, sd_density_mid, sd_density_small + rremove("x.text"), 
          labels = c("Large Cap", "Mid Cap", "Small Cap"), font.label = list(size = 10, color = "black"),
          ncol = 3, nrow = 1)
annotate_figure(sdplot, top = text_grob("Rolling 12-Month Risk Comparison (past 10 Years)", 
               color = "black", face = "bold", size = 11))
\end{verbatim}

The `average' volatility for each index was then calculated to compare
to the rolling standard deviation calculation.

```\{r STD\} largevol \textless-
StdDev(l\_caps\(Daily_return) midvol <- StdDev(m_caps\)Daily\_return)
smallvol \textless- StdDev(s\_caps\$Daily\_return)

cat(``Large volatility:'', largevol, ``\n'') cat(``Mid volatility:'',
midvol, ``\n'') cat(``Small volatility:'', smallvol)

\begin{verbatim}
The results were manually tabulated as follows. 

\begin{table}[h]
\begin{center}
    \begin{tabular}{| c | c |}
    \hline
         & Standard deviation \\
        \hline
        Large cap & 0.01298498 \\
        Mid cap & 0.01039368 \\
        Small cap & 0.00801369 \\
        \hline
    \end{tabular}
    \caption{Standard deviation}
    \label{tab:SD}
\end{center}
\end{table}

Now, the volatility skewness of each index was calculated. This is a ratio of upside variance to downside variance. 

```{r vol}
vol_returns <- list(ljoin, mjoin, sjoin) |>
    reduce(full_join, by='date') |> 
    tbl_xts()
VolatilitySkewness(vol_returns, MAR = 0, stat = c("volatility"))
\end{verbatim}

The results were again manually tabulated.

\begin{table}[h]
\begin{center}
    \begin{tabular}{| c | c | c | c |}
    \hline
         & Large cap & Mid cap & Small cap \\
        \hline
        Volatility skewness & 1.069731 & 0.9739739 & 1.379238 \\
        \hline
    \end{tabular}
    \caption{Volatility skewness}
    \label{tab:VS}
\end{center}
\end{table}

\hypertarget{correlation}{%
\section{Correlation}\label{correlation}}

The correlation calculation was rather involved. I first created an
index to determine the degree of correlation between all stocks within
each respective index. After those datafiles were obtained, separate
density plots were produced using ggplot2. The resulting graphs were
then combined and were used in the paper.

```\{r Correlation\} \#Function to determine the correlation between
constituent shares within each index. correlationreturn \textless-
function(x)\{ index \textless- x index \textless- index{[},-c(5:10){]}
index\(Tickers <- gsub(" SJ|Equity","", index\)Tickers) Total \textless-
index \textbar\textgreater{} mutate(Contribution =
coalesce(Contribution, 0)) \textbar\textgreater{} group\_by(Tickers)
\textbar\textgreater{} mutate(Cumulative\_contribution = cumprod(1 +
Contribution)) Final \textless- Total{[},c(1,2,3){]} wide \textless-
spread(Final, Tickers, Return) wide1 \textless- as.data.frame(wide)
wide2 \textless- wide1{[}-c(1:1750),{]} \#wide1{[}is.na(wide1){]} = 0
\#Made NAs equal to zero so that rowSums function would work xts
\textless- wide2{[},c(2:86){]} \textbar\textgreater{} xts(order.by =
as.Date(wide2{[} , 1{]}, ``\%Y-\%m-\%d'')) y \textless- rollapply(xts,
width=365, cor, fill = NA, by.column=FALSE) paste\_ \textless-
function(\ldots) paste(\ldots, sep = ``\emph{``) names\_mat \textless-
do.call(''outer'', list(names(xts), names(xts), paste})) names(y)
\textless- names\_mat
\#y\(Average_correlation <- rowMeans(y, na.rm=TRUE)  e <- y |>  xts_tbl()  f <- rbind(e, sapply(e, function(x) if(is.numeric(x)) mean(x, na.rm = TRUE) else ""))  last <- f[nrow(f),]  long <- last |>  gather(Stocks, Correlation, -date)  long1 <- long[,-1]  long2 <- arrange(long1, Correlation)  long3 <- long2[!duplicated(long2\)Correlation),
{]} long3\(Correlation <- as.numeric(as.character(long3\)Correlation))
\#long4 \textless- long3{[}!is.nan(long3){]} \textbar\textgreater{} \#
filter(long3\$Correlation != 1) \#long3 = as.numeric(long2{[},2{]})
return(long3) \}

\#Large Cap Index correalation dataframe large\_corr \textless-
correlationreturn(J200) \#Large Cap Index density plot of correlations
large\_corr1 \textless- large\_corr{[}-c(1493:1494),{]} plot\_largecorr
\textless- ggplot(large\_corr1, aes(x = Correlation)) +
geom\_density(color=``darkred'', fill=``red'') +
geom\_vline(aes(xintercept = mean(Correlation)), linetype = ``dashed'',
size = 0.6) + coord\_flip() + scale\_x\_continuous(limits = c(0, 0.5),
labels = scales::percent\_format(scale = 100)) +
theme(axis.text.x=element\_blank(), axis.ticks.x=element\_blank(),
axis.title.x = element\_blank(), axis.title.y = element\_blank(),
panel.grid.major = element\_blank(), panel.grid.minor =
element\_blank(), panel.background = element\_blank()) +
theme(legend.position=``none'')

\#Mid Cap Index correalation dataframe mid\_corr \textless-
correlationreturn(J201) \#Mid Cap Index density plot of correlations
mid\_corr1 \textless- mid\_corr{[}-c(1044:1045),{]} plot\_midcorr
\textless- ggplot(mid\_corr1, aes(x = Correlation)) +
geom\_density(color=``darkblue'', fill=``lightblue'') +
geom\_vline(aes(xintercept = mean(Correlation)), linetype = ``dashed'',
size = 0.6) + coord\_flip() + scale\_x\_continuous(limits = c(0, 0.5),
labels = scales::percent\_format(scale = 100)) +
theme(axis.text.x=element\_blank(), axis.ticks.x=element\_blank(),
axis.title.x = element\_blank(), axis.title.y = element\_blank(),
panel.grid.major = element\_blank(), panel.grid.minor =
element\_blank(), panel.background = element\_blank()) +
theme(legend.position=``none'')

\#Small Cap Index correalation dataframe small\_corr \textless-
correlationreturn(J202) \#Small Cap Index density plot of correlations
small\_corr1 \textless- small\_corr{[}-c(721:722),{]} plot\_smallcorr
\textless- ggplot(small\_corr1, aes(x = Correlation)) +
geom\_density(color=``darkgreen'', fill=``lightgreen'') +
geom\_vline(aes(xintercept = mean(Correlation)), linetype = ``dashed'',
size = 0.6) + coord\_flip() + scale\_x\_continuous(limits = c(0, 0.5),
labels = scales::percent\_format(scale = 100)) +
theme(axis.text.x=element\_blank(), axis.ticks.x=element\_blank(),
axis.title.x = element\_blank(), axis.title.y = element\_blank(),
panel.grid.major = element\_blank(), panel.grid.minor =
element\_blank(), panel.background = element\_blank()) +
theme(legend.position=``none'')

\begin{verbatim}
The graphs were combined and plotted. 

```{r Figure11}
corrplot <- ggarrange(plot_largecorr, plot_midcorr, plot_smallcorr + rremove("x.text"), 
          labels = c("Large Cap", "Mid Cap", "Small Cap"), font.label = list(size = 10, color = "black"),
          ncol = 3, nrow = 1)
annotate_figure(corrplot, top = text_grob("Rolling 12-Month Correlation Comparison (past 10 Years)", 
               color = "black", face = "bold", size = 11))
\end{verbatim}

\end{document}
