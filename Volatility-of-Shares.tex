\documentclass[11pt,preprint, authoryear]{elsarticle}

\usepackage{lmodern}
%%%% My spacing
\usepackage{setspace}
\setstretch{1.2}
\DeclareMathSizes{12}{14}{10}{10}

% Wrap around which gives all figures included the [H] command, or places it "here". This can be tedious to code in Rmarkdown.
\usepackage{float}
\let\origfigure\figure
\let\endorigfigure\endfigure
\renewenvironment{figure}[1][2] {
    \expandafter\origfigure\expandafter[H]
} {
    \endorigfigure
}

\let\origtable\table
\let\endorigtable\endtable
\renewenvironment{table}[1][2] {
    \expandafter\origtable\expandafter[H]
} {
    \endorigtable
}


\usepackage{ifxetex,ifluatex}
\usepackage{fixltx2e} % provides \textsubscript
\ifnum 0\ifxetex 1\fi\ifluatex 1\fi=0 % if pdftex
  \usepackage[T1]{fontenc}
  \usepackage[utf8]{inputenc}
\else % if luatex or xelatex
  \ifxetex
    \usepackage{mathspec}
    \usepackage{xltxtra,xunicode}
  \else
    \usepackage{fontspec}
  \fi
  \defaultfontfeatures{Mapping=tex-text,Scale=MatchLowercase}
  \newcommand{\euro}{€}
\fi

\usepackage{amssymb, amsmath, amsthm, amsfonts}

\def\bibsection{\section*{References}} %%% Make "References" appear before bibliography


\usepackage[round]{natbib}

\usepackage{longtable}
\usepackage[margin=2.3cm,bottom=2cm,top=2.5cm, includefoot]{geometry}
\usepackage{fancyhdr}
\usepackage[bottom, hang, flushmargin]{footmisc}
\usepackage{graphicx}
\numberwithin{equation}{section}
\numberwithin{figure}{section}
\numberwithin{table}{section}
\setlength{\parindent}{0cm}
\setlength{\parskip}{1.3ex plus 0.5ex minus 0.3ex}
\usepackage{textcomp}
\renewcommand{\headrulewidth}{0.2pt}
\renewcommand{\footrulewidth}{0.3pt}

\usepackage{array}
\newcolumntype{x}[1]{>{\centering\arraybackslash\hspace{0pt}}p{#1}}

%%%%  Remove the "preprint submitted to" part. Don't worry about this either, it just looks better without it:
\makeatletter
\def\ps@pprintTitle{%
  \let\@oddhead\@empty
  \let\@evenhead\@empty
  \let\@oddfoot\@empty
  \let\@evenfoot\@oddfoot
}
\makeatother

 \def\tightlist{} % This allows for subbullets!

\usepackage{hyperref}
\hypersetup{breaklinks=true,
            bookmarks=true,
            colorlinks=true,
            citecolor=blue,
            urlcolor=blue,
            linkcolor=blue,
            pdfborder={0 0 0}}


% The following packages allow huxtable to work:
\usepackage{siunitx}
\usepackage{multirow}
\usepackage{hhline}
\usepackage{calc}
\usepackage{tabularx}
\usepackage{booktabs}
\usepackage{caption}


\newenvironment{columns}[1][]{}{}

\newenvironment{column}[1]{\begin{minipage}{#1}\ignorespaces}{%
\end{minipage}
\ifhmode\unskip\fi
\aftergroup\useignorespacesandallpars}

\def\useignorespacesandallpars#1\ignorespaces\fi{%
#1\fi\ignorespacesandallpars}

\makeatletter
\def\ignorespacesandallpars{%
  \@ifnextchar\par
    {\expandafter\ignorespacesandallpars\@gobble}%
    {}%
}
\makeatother

\newlength{\cslhangindent}
\setlength{\cslhangindent}{1.5em}
\newenvironment{CSLReferences}%
  {\setlength{\parindent}{0pt}%
  \everypar{\setlength{\hangindent}{\cslhangindent}}\ignorespaces}%
  {\par}


\urlstyle{same}  % don't use monospace font for urls
\setlength{\parindent}{0pt}
\setlength{\parskip}{6pt plus 2pt minus 1pt}
\setlength{\emergencystretch}{3em}  % prevent overfull lines
\setcounter{secnumdepth}{5}

%%% Use protect on footnotes to avoid problems with footnotes in titles
\let\rmarkdownfootnote\footnote%
\def\footnote{\protect\rmarkdownfootnote}
\IfFileExists{upquote.sty}{\usepackage{upquote}}{}

%%% Include extra packages specified by user

%%% Hard setting column skips for reports - this ensures greater consistency and control over the length settings in the document.
%% page layout
%% paragraphs
\setlength{\baselineskip}{12pt plus 0pt minus 0pt}
\setlength{\parskip}{12pt plus 0pt minus 0pt}
\setlength{\parindent}{0pt plus 0pt minus 0pt}
%% floats
\setlength{\floatsep}{12pt plus 0 pt minus 0pt}
\setlength{\textfloatsep}{20pt plus 0pt minus 0pt}
\setlength{\intextsep}{14pt plus 0pt minus 0pt}
\setlength{\dbltextfloatsep}{20pt plus 0pt minus 0pt}
\setlength{\dblfloatsep}{14pt plus 0pt minus 0pt}
%% maths
\setlength{\abovedisplayskip}{12pt plus 0pt minus 0pt}
\setlength{\belowdisplayskip}{12pt plus 0pt minus 0pt}
%% lists
\setlength{\topsep}{10pt plus 0pt minus 0pt}
\setlength{\partopsep}{3pt plus 0pt minus 0pt}
\setlength{\itemsep}{5pt plus 0pt minus 0pt}
\setlength{\labelsep}{8mm plus 0mm minus 0mm}
\setlength{\parsep}{\the\parskip}
\setlength{\listparindent}{\the\parindent}
%% verbatim
\setlength{\fboxsep}{5pt plus 0pt minus 0pt}



\begin{document}



%titlepage
\thispagestyle{empty}
\begin{center}
\begin{minipage}{0.75\linewidth}
    \centering
%Entry1
    {\uppercase{\huge Analysing the volatility of large, mid and small
cap shares in South Africa\par}}
    \vspace{2cm}
%Author's name
    {\LARGE \textbf{Joshua Strydom (20718284)}\par}
    \vspace{1cm}
%University logo
\begin{center}
    \includegraphics[width=0.3\linewidth]{Tex/Stellenbosch.jpeg}
\end{center}
\vspace{1cm}
%Supervisor's Details
\begin{center}
    {\Large \par}
    \vspace{1cm}
%Degree
    {\large Stellenbosch University\par}
    \vspace{1cm}
%Institution
    {\large January 2023\par}
    \vspace{1cm}
%Date
    {\large }
%More
    {\normalsize }
%More
    {\normalsize }
\end{center}
\end{minipage}
\end{center}
\clearpage


\begin{frontmatter}  %

\title{Analysing the volatility of large, mid and small cap shares in
South Africa}

% Set to FALSE if wanting to remove title (for submission)




\author[Add1]{Joshua Strydom\footnote{\textbf{Contributions:}
  \newline \emph{The authors would like to thank no institution for
  money donated to this project. Thank you sincerely.}}}
\ead{20718284@sun.ac.za}





\address[Add1]{Stellenbosch University, Stellenbosch, South Africa}



\vspace{1cm}


\begin{keyword}
\footnotesize{
Volatility \sep ALSI \sep Standard deviation \\
\vspace{0.3cm}
}
\footnotesize{
\textit{JEL classification} L250 \sep L100
}
\end{keyword}



\vspace{0.5cm}

\end{frontmatter}



%________________________
% Header and Footers
%%%%%%%%%%%%%%%%%%%%%%%%%%%%%%%%%
\pagestyle{fancy}
\chead{}
\rhead{}
\lfoot{}
\rfoot{\footnotesize Page \thepage}
\lhead{}
%\rfoot{\footnotesize Page \thepage } % "e.g. Page 2"
\cfoot{}

%\setlength\headheight{30pt}
%%%%%%%%%%%%%%%%%%%%%%%%%%%%%%%%%
%________________________

\headsep 35pt % So that header does not go over title




\hypertarget{introduction}{%
\section{\texorpdfstring{Introduction
\label{Introduction}}{Introduction }}\label{introduction}}

Generally one would think that small cap stocks represent companies that
have less established business models and less predictable revenue
streams. This might lead one to believe that the prices of such stocks
fluctuate to a greater degree than a more established, larger company.
This would result in a less predictable return profile and hence,
greater risk. The aim of this paper is to asses whether small cap stocks
have a higher or lower level of volatility than large and mid cap stocks
in South Africa.

Three separate indexes (Large, Mid and Small) were established in order
to conduct the analysis over the 2005 to 2022 period. The Global
Financial Crisis (GFC) and the COVID-19 pandemic had obvious
consequences for the three indexes. The objective was both to determine
which of the indexes were most affected by these times of crisis as well
as to determine which of the indexes recovered fastest. It is found that
the composition and the degree of correlation of an index has profound
consequences for the performance of the respective index. As will be
divulged, the Small Cap Index generated the lowest level of volatility
when compared to the Large Cap and Mid Cap Indexes. The Small Cap
Index's returns, standard deviations and degree of correlation all
contributed to the resultant observation. The low degree of correlation
of the Small Cap Index allowed for it be be well-diversified and best
suited to weather an economic crisis in South Africa.

\hypertarget{index-analysis}{%
\section{\texorpdfstring{Index Analysis
\label{Index}}{Index Analysis }}\label{index-analysis}}

The analysis of volatility is based on three indexes of the Johannesburg
Securities Exchange (JSE). Data is attained by filtering out the FTSE
All Share Index (ALSI). The large cap index is based on the FTSE/JSE Top
40 Index (J200), the mid cap index is based on the FTSE/JSE Mid Cap
Index (J201) and the small cap index is based on the FTSE/JSE Small Cap
Index (J202). The derived returns for each of the separate indexes are
calculated as the product of a stocks daily return and the assigned
weight in the respective index. For the large and small cap indexes,
data from 2005 to 2022 is available. Unfortunately for the mid cap
index, data is only available from 2016 onward.

\hypertarget{drawdowns}{%
\subsection{Drawdowns}\label{drawdowns}}

A drawdown chart is used to illustrate the worst periods of time for a
portfolio under analysis. The graphs are important for measuring the
historical risk of an investment as they display the peak-to-trough
decline in each of the indexes value. A drawdown chart is essentially a
measure of downside volatility.

The large cap index in Figure\ref{Figure1} experienced its most drastic
drawdown between 2008 and 2009 but most drawdowns were in the range of
between 0 and -0.2. This is not surprising as the large cap index has a
relatively large international exposure and hence, the aftermath of the
Global Financial Crisis (GFC) would have been heavy hitting on the
returns of the index. The next major drawdown occured in the period
during the COVID-19 pandemic. The large cap index is thus highly
reactive to times of economic crisis. This is no surprise.

\begin{figure}[H]

{\centering \includegraphics{Volatility-of-Shares_files/figure-latex/Figure1-1} 

}

\caption{Large Cap drawdown chart \label{Figure1}}\label{fig:Figure1}
\end{figure}

It is unfortunate that data could not be obtained for the mid cap index
for the full sample period. Most drawdowns for this short sample of the
mid cap index were in the range of between 0 and -0.2. During the
COVID-19 pandemic, the mid cap index responded very similarly to the
large cap index. I attribute this common drawdown to the fact that the
pandemic was not selective on which stocks within an index would be more
or less affected. The pandemic was a global crisis and had nothing to do
with an index's amount of international exposure. The GFC would have had
more of an impact on an index with high international exposure compared
to one with lower international exposure.

\begin{figure}[H]

{\centering \includegraphics{Volatility-of-Shares_files/figure-latex/Figure2-1} 

}

\caption{Mid Cap drawdown chart \label{Figure2}}\label{fig:Figure2}
\end{figure}

As illustrated in Figure \ref{Figure3}, the max drawdown for the small
cap index occurred in the aftermath of the GFC. This drawdown was of a
smaller magnitude and lasted for a shorter period when compared to the
drawdown of the large cap index for the same period. The small cap index
was not immune to the effects of the COVID-19 pandemic and so
experienced a drawdown of a similar magnitude to both the large and mid
cap indexes.

\begin{figure}[H]

{\centering \includegraphics{Volatility-of-Shares_files/figure-latex/Figure3-1} 

}

\caption{Small Cap drawdown chart \label{Figure3}}\label{fig:Figure3}
\end{figure}

\hypertarget{sector-weightings}{%
\subsection{Sector weightings}\label{sector-weightings}}

Table \ref{tab:Mean} below provides a depiction of the mean of daily
returns for each sector over the entire period. The calculation was
based solely on a stocks (within a certain sector) return and did not
account for any weighting within an index. The `Industrials' sector
clearly generates, on average, the highest mean returns by a large
magnitude. The `Financials' sector generated the second highest mean
daily returns while the `Property' and `Resources' sectors trailed far
behind. The natural goal of an index would be to optimise its return
profile. This would involve choosing a group of stocks that generate
high returns but are not correlated (or have a low correlation). Hedging
against downside returns could even be considered as important as
searching for upside returns.

\begin{table}[h]
\begin{center}
    \begin{tabular}{| c | c |}
    \hline
        Sector & Mean \\
        \hline
        Financials & 0.16530 \\
        Industrials & 1.342 \\
        Property & 0.009539 \\
        Resources & 0.07961 \\
        \hline
    \end{tabular}
    \caption{Mean daily returns}
    \label{tab:Mean}
\end{center}
\end{table}

As can be seen in \ref{Figure4}, the large cap index has historically
neglected the `Property' sector and has been dominated mostly by the
`Industrial' and `Resource' sectors. In earlier periods of the sample
(2005 to 2012) `Resources' dominated `Industrials' but the opposite
became true in later periods. The `Financials' sector seems to have
maintained a relatively constant weighting in the large cap index. The
weightings in the later portion of the period do not sum to one. This is
the result of a few mid cap stocks being filtered out of the large cap
index yet they formed part of the FTSE/JSE Top 40 Index.

\begin{figure}[H]

{\centering \includegraphics{Volatility-of-Shares_files/figure-latex/Figure4-1} 

}

\caption{Sector weights for Large Cap index \label{Figure4}}\label{fig:Figure4}
\end{figure}

The `Industrials' sector also dominated the weighting of the mid cap
index. Figure \ref{Figure5} displays a consistent domination of the
`Industrials' sector for the index. The `Financial' sector was assigned
the second highest weighting in the latter portion of the period at
hand, surpassing the early dominance of the `Resources' and `Property'
sector. The mid cap index assigned a noticeably greater weight to the
`Property' sector and a lower weight to the `Resources' sector when
compared to the large cap index for the latter portion of the full
period under analysis.

\begin{figure}[H]

{\centering \includegraphics{Volatility-of-Shares_files/figure-latex/Figure5-1} 

}

\caption{Sector weights for Mid Cap index \label{Figure5}}\label{fig:Figure5}
\end{figure}

When compared to the large and mid cap indexes, the small cap index
placed the lowest weighting on the `Financial' sector. The `Property'
sector in the small cap index was allocated the most weight out of all
the indexes from around 2013 to 2022. The `Resources' sector was not
favoured and was assigned the smallest weighting in the small cap index
for most of the period.\\
The `Industrials' sector once again dominated the weighting of the
index, with approximately 50\% of the small cap index being dominated by
the sector for the full period.

\begin{figure}[H]

{\centering \includegraphics{Volatility-of-Shares_files/figure-latex/Figure6-1} 

}

\caption{Sector weights for Small Cap index \label{Figure6}}\label{fig:Figure6}
\end{figure}

\hypertarget{annualised-returns}{%
\section{\texorpdfstring{Annualised returns
\label{Returns}}{Annualised returns }}\label{annualised-returns}}

The idea behind annualising returns is to be able to make direct
comparisons of returns across different time periods. A rolling
annualised return calculation was conducted in order to provide a means
for actual performance comparison for the indexes at hand. As can be
seen in Figure \ref{Figure7} below, the small cap index consistently
outperforms both the large and mid cap indexes. An elementary analysis
based on the figure of how large and small cap shares responded to the
aftermath of the Global Financial Crisis of 2007/2008 yields interesting
results. The small cap index seems to have had enjoyed increasingly
positive returns subsequent to 2009 which may be due to the fact that
the index had a low international exposure. This point is reaffirmed by
Figures \ref{Figure1} and \ref{Figure3} in that the drawdown of the
large cap index was more prolonged than the small cap drawdown even
though the magnitude of drawdowns were similar. Although the returns for
the large cap index never turned negative, the `level of positivity' of
returns diminished subsequent to the crisis. The annualised returns for
the large and small indexes were most similar between 2013 and 2015.

The mid cap index outperformed the large cap index from 2016 to
approximately half way through 2020. Thereafter, the large cap index
outperformed the mid cap index. Thus, a closer look at the pandemic
period (2020 - 2022) seemed necessary. The particular point of interest
is when the large and mid cap return lines intersected. Figures
\ref{Figure1} and \ref{Figure2} provide insight on how the index
compositions changed during this pandemic period. The large cap index
assigned a greater weight to the `Resources' sector, kept the weighting
of the `Industrials' sector relatively constant and down-weighted both
the `Financials' and the `Property' sectors. The mid cap index , on the
other hand, assigned a greater weight to the `Financials' sector and
down-weighted the other three sectors. The `Financial' sector in South
Africa absorbed heavy blows during and subsequent to the pandemic while
the `Resources' sector experienced a less dramatic downfall. This
weighting discrepancy/mistake, as I see it, resulted in the
unsatisfactory performance of the mid cap index. The small cap index
assigned weights similarly to the large cap index with the exception of
an increased weighting for the `Property' sector as opposed to a
decreased one.

It is important to note that the small cap index consistently
outperformed both the large and mid cap indexes throughout the period of
analysis.

\begin{figure}[H]

{\centering \includegraphics{Volatility-of-Shares_files/figure-latex/Figure7-1} 

}

\caption{Rolling Annualised Returns of the Indexes \label{Figure7}}\label{fig:Figure7}
\end{figure}

\begin{figure}[H]

{\centering \includegraphics{Volatility-of-Shares_files/figure-latex/Figure8-1} 

}

\caption{Rolling 12-Month Returns Comparison (past 10 Years) \label{Figure8}}\label{fig:Figure8}
\end{figure}

Both the Mid- and Small Cap indexes provide high potential upside
returns. The horizontal lines running through each graph in Figure
\ref{Figure8} represent the respective means of the rolling 12-month
returns over the past 10 years. The Small Cap Index's mean return is
significantly higher than both the Mid Cap Index and Large Cap Index
mean returns. Although the Mid Cap Index mean return is below the
0-line, most returns are distributed above the 0-line. There is thus a
higher likelihood for upside returns than downside ones.

The return distribution of the Small Cap Index, in particular, is
significantly more skewed to positive returns than the other indexes. It
thus provides the greatest potential for upside gains.

\hypertarget{rolling-standard-deviation}{%
\section{\texorpdfstring{Rolling Standard deviation
\label{Standard deviation}}{Rolling Standard deviation }}\label{rolling-standard-deviation}}

\begin{figure}[H]

{\centering \includegraphics{Volatility-of-Shares_files/figure-latex/Figure9-1} 

}

\caption{Rolling Annualised Standard Deviation of the Indexes \label{Figure9}}\label{fig:Figure9}
\end{figure}

Standard deviation in this analysis is a used as a measure of risk.
Returns for each index are annualised so that standard deviation can be
compared over different time periods. It is apparent from Figure
\ref{Figure9} that between 2005 and 2015 the Large Cap Index is on
average epitomised by a larger standard deviation than the Small Cap
Index. It should now be of no surprise that the standard deviation of
the Large Cap Index spikes, as a result of the GFC, in 2009 more than
that of the Small Cap Index. The standard deviation of all indexes spike
in 2016 and halfway through 2020. The spike in the Small Cap Index,
however, exceeds the standard deviation spikes of both the Large Cap and
Mid Cap indexes during these time stamps. Following the 2020 spike, the
Small Cap Index quickly reverts back to pre-pandemic levels of standard
deviation while the Mid Cap and Large Cap Indexes take a little longer
to revert back. By the end of the period of analysis, the standard
deviations of the three indexes are reasonably close together with the
Small Cap Index having the lowest and the Large Cap Index having the
highest.

\begin{figure}[H]

{\centering \includegraphics{Volatility-of-Shares_files/figure-latex/Figure10-1} 

}

\caption{Rolling 12-month Risk Comparison (past 10 Years) \label{Figure10}}\label{fig:Figure10}
\end{figure}

The high potential for upside returns in the Small Cap Index does not
necessarily come at a higher portfolio risk. The distribution of
materialized risk over time shows that the Small Cap Index delivered
lower aggregate volatility on the whole. The stability of the index can
be attributed to the lower relative international exposure compared to
both the Large and Mid Cap Indexes.

The Mid Cap Index seems to have delivered the highest aggregate
volatility on the whole. This result is likely to have been influenced
by the lack of data. Although the Large Cap Index's distribution of
materialized risk over time is concentrated between the 2\% and 3\%
regions, its mean standard deviation is roughly equal to the mean
standard deviation of the Mid Cap Index.

\begin{table}[h]
\begin{center}
    \begin{tabular}{| c | c |}
    \hline
         & Standard deviation \\
        \hline
        Large cap & 0.01298498 \\
        Mid cap & 0.01039368 \\
        Small cap & 0.00801369 \\
        \hline
    \end{tabular}
    \caption{Standard deviation}
    \label{tab:SD}
\end{center}
\end{table}

Table \ref{tab:SD} displays the overall standard deviation for each
index over their respective sample periods. The Large Cap Index is
associated with the highest standard deviation and as such, the highest
level of risk.

\begin{table}[h]
\begin{center}
    \begin{tabular}{| c | c | c | c |}
    \hline
         & Large cap & Mid cap & Small cap \\
        \hline
        Volatility skewness & 1.069731 & 0.9739739 & 1.379238 \\
        \hline
    \end{tabular}
    \caption{Volatility skewness}
    \label{tab:VS}
\end{center}
\end{table}

Table \ref{tab:VS} displays a measure of volatility skewness. It is a
ratio of upside variance to downside variance.

\hypertarget{correlation}{%
\section{\texorpdfstring{Correlation
\label{Correlation}}{Correlation }}\label{correlation}}

\begin{figure}[H]

{\centering \includegraphics{Volatility-of-Shares_files/figure-latex/Figure11-1} 

}

\caption{Rolling Correlation Comparison of the Indexes \label{Figure11}}\label{fig:Figure11}
\end{figure}

Figure \ref{Figure11} shows the degree to which stocks within each of
the Large, Mid and Small Cap indexes are correlated. The correlations
are measured on a rolling 12-month basis, with the distributions showing
the extent to which correlation varies in each.

Within the Mid and Small Cap Indexes, individual correlations are lower.
So, in addition to the high potential upside returns and lower
associated risk, the Small Cap Index also provides increased
diversification. Although the Mid Cap Index could be regarded as
relatively risky, it is more diversified than the Large Cap Index. This
associated risk thus has to be derived from somewhere other than
correlation between individual stocks within the portfolio.

The Large Cap Index has the highest degree of correlation between its
constituent stocks. The Large Cap Index is dominated by stocks that have
a high degree of international exposure. Changes in global conditions
and economic activity affect these stocks directly and thus leads to a
higher level of correlation. A high degree of correlation negatively
impacts any chances of diversification.

\hypertarget{conclusion}{%
\section{\texorpdfstring{Conclusion
\label{Conclusion}}{Conclusion }}\label{conclusion}}

The aim of this paper was to compare the differing levels of volatility
in large, mid and small cap stocks. The Small Cap Index was not immune
to the consequences of the GFC of 2007/2008 but was not nearly as
adversely affected when compared to the Large Cap Index. The high degree
of international exposure and hence, high degree of correlation did not
allow for the Large Cap Index to be sufficiently diversified to combat
the effects of the GFC. The pandemic of 2020/2021 had a more
all-encompassing consequence for stocks as a whole in South Africa. The
consequences were not isolated to a few sectors of the economy but
rather they adversely impacted most. The lower degree of correlation of
both the Small and Mid Cap Indexes assisted these indexes during this
time. The composition of the Mid Cap Index did not, however, do the
annualised return of the index any favours. The Large Cap Index was
better composed and so bounced back better than the Mid Cap Index.

The Small Cap Index has the greatest potential for upside returns, has
the lowest associated risk and has the lowest degree of correlation. The
return profile of the index was thus able to be optimised via
diversification. It can thus to said with confidence that the Small Cap
Index is the most stable in South Africa.

\newpage

\hypertarget{references}{%
\section*{References}\label{references}}
\addcontentsline{toc}{section}{References}

\hypertarget{refs}{}
\begin{CSLReferences}{0}{0}
\end{CSLReferences}

\hypertarget{appendix}{%
\section*{Appendix}\label{appendix}}
\addcontentsline{toc}{section}{Appendix}

\hypertarget{appendix-a}{%
\subsection*{Appendix A}\label{appendix-a}}
\addcontentsline{toc}{subsection}{Appendix A}

Some appendix information here

\hypertarget{appendix-b}{%
\subsection*{Appendix B}\label{appendix-b}}
\addcontentsline{toc}{subsection}{Appendix B}

\bibliography{Tex/ref}





\end{document}
