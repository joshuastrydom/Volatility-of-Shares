\documentclass[11pt,preprint, authoryear]{elsarticle}

\usepackage{lmodern}
%%%% My spacing
\usepackage{setspace}
\setstretch{1.2}
\DeclareMathSizes{12}{14}{10}{10}

% Wrap around which gives all figures included the [H] command, or places it "here". This can be tedious to code in Rmarkdown.
\usepackage{float}
\let\origfigure\figure
\let\endorigfigure\endfigure
\renewenvironment{figure}[1][2] {
    \expandafter\origfigure\expandafter[H]
} {
    \endorigfigure
}

\let\origtable\table
\let\endorigtable\endtable
\renewenvironment{table}[1][2] {
    \expandafter\origtable\expandafter[H]
} {
    \endorigtable
}


\usepackage{ifxetex,ifluatex}
\usepackage{fixltx2e} % provides \textsubscript
\ifnum 0\ifxetex 1\fi\ifluatex 1\fi=0 % if pdftex
  \usepackage[T1]{fontenc}
  \usepackage[utf8]{inputenc}
\else % if luatex or xelatex
  \ifxetex
    \usepackage{mathspec}
    \usepackage{xltxtra,xunicode}
  \else
    \usepackage{fontspec}
  \fi
  \defaultfontfeatures{Mapping=tex-text,Scale=MatchLowercase}
  \newcommand{\euro}{€}
\fi

\usepackage{amssymb, amsmath, amsthm, amsfonts}

\def\bibsection{\section*{References}} %%% Make "References" appear before bibliography


\usepackage[round]{natbib}

\usepackage{longtable}
\usepackage[margin=2.3cm,bottom=2cm,top=2.5cm, includefoot]{geometry}
\usepackage{fancyhdr}
\usepackage[bottom, hang, flushmargin]{footmisc}
\usepackage{graphicx}
\numberwithin{equation}{section}
\numberwithin{figure}{section}
\numberwithin{table}{section}
\setlength{\parindent}{0cm}
\setlength{\parskip}{1.3ex plus 0.5ex minus 0.3ex}
\usepackage{textcomp}
\renewcommand{\headrulewidth}{0.2pt}
\renewcommand{\footrulewidth}{0.3pt}

\usepackage{array}
\newcolumntype{x}[1]{>{\centering\arraybackslash\hspace{0pt}}p{#1}}

%%%%  Remove the "preprint submitted to" part. Don't worry about this either, it just looks better without it:
\makeatletter
\def\ps@pprintTitle{%
  \let\@oddhead\@empty
  \let\@evenhead\@empty
  \let\@oddfoot\@empty
  \let\@evenfoot\@oddfoot
}
\makeatother

 \def\tightlist{} % This allows for subbullets!

\usepackage{hyperref}
\hypersetup{breaklinks=true,
            bookmarks=true,
            colorlinks=true,
            citecolor=blue,
            urlcolor=blue,
            linkcolor=blue,
            pdfborder={0 0 0}}


% The following packages allow huxtable to work:
\usepackage{siunitx}
\usepackage{multirow}
\usepackage{hhline}
\usepackage{calc}
\usepackage{tabularx}
\usepackage{booktabs}
\usepackage{caption}


\newenvironment{columns}[1][]{}{}

\newenvironment{column}[1]{\begin{minipage}{#1}\ignorespaces}{%
\end{minipage}
\ifhmode\unskip\fi
\aftergroup\useignorespacesandallpars}

\def\useignorespacesandallpars#1\ignorespaces\fi{%
#1\fi\ignorespacesandallpars}

\makeatletter
\def\ignorespacesandallpars{%
  \@ifnextchar\par
    {\expandafter\ignorespacesandallpars\@gobble}%
    {}%
}
\makeatother

\newlength{\cslhangindent}
\setlength{\cslhangindent}{1.5em}
\newenvironment{CSLReferences}%
  {\setlength{\parindent}{0pt}%
  \everypar{\setlength{\hangindent}{\cslhangindent}}\ignorespaces}%
  {\par}


\urlstyle{same}  % don't use monospace font for urls
\setlength{\parindent}{0pt}
\setlength{\parskip}{6pt plus 2pt minus 1pt}
\setlength{\emergencystretch}{3em}  % prevent overfull lines
\setcounter{secnumdepth}{5}

%%% Use protect on footnotes to avoid problems with footnotes in titles
\let\rmarkdownfootnote\footnote%
\def\footnote{\protect\rmarkdownfootnote}
\IfFileExists{upquote.sty}{\usepackage{upquote}}{}

%%% Include extra packages specified by user

%%% Hard setting column skips for reports - this ensures greater consistency and control over the length settings in the document.
%% page layout
%% paragraphs
\setlength{\baselineskip}{12pt plus 0pt minus 0pt}
\setlength{\parskip}{12pt plus 0pt minus 0pt}
\setlength{\parindent}{0pt plus 0pt minus 0pt}
%% floats
\setlength{\floatsep}{12pt plus 0 pt minus 0pt}
\setlength{\textfloatsep}{20pt plus 0pt minus 0pt}
\setlength{\intextsep}{14pt plus 0pt minus 0pt}
\setlength{\dbltextfloatsep}{20pt plus 0pt minus 0pt}
\setlength{\dblfloatsep}{14pt plus 0pt minus 0pt}
%% maths
\setlength{\abovedisplayskip}{12pt plus 0pt minus 0pt}
\setlength{\belowdisplayskip}{12pt plus 0pt minus 0pt}
%% lists
\setlength{\topsep}{10pt plus 0pt minus 0pt}
\setlength{\partopsep}{3pt plus 0pt minus 0pt}
\setlength{\itemsep}{5pt plus 0pt minus 0pt}
\setlength{\labelsep}{8mm plus 0mm minus 0mm}
\setlength{\parsep}{\the\parskip}
\setlength{\listparindent}{\the\parindent}
%% verbatim
\setlength{\fboxsep}{5pt plus 0pt minus 0pt}



\begin{document}



%titlepage
\thispagestyle{empty}
\begin{center}
\begin{minipage}{0.75\linewidth}
    \centering
%Entry1
    {\uppercase{\huge Analysing the volatility of large, mid and small
cap shares in South Africa\par}}
    \vspace{2cm}
%Author's name
    {\LARGE \textbf{Joshua Strydom}\par}
    \vspace{1cm}
%University logo
\begin{center}
    \includegraphics[width=0.3\linewidth]{Tex/Stellenbosch.jpeg}
\end{center}
\vspace{1cm}
%Supervisor's Details
\begin{center}
    {\Large \par}
    \vspace{1cm}
%Degree
    {\large Stellenbosch University\par}
    \vspace{1cm}
%Institution
    {\large January 2023\par}
    \vspace{1cm}
%Date
    {\large }
%More
    {\normalsize }
%More
    {\normalsize }
\end{center}
\end{minipage}
\end{center}
\clearpage


\begin{frontmatter}  %

\title{Analysing the volatility of large, mid and small cap shares in
South Africa}

% Set to FALSE if wanting to remove title (for submission)




\author[Add1]{Joshua Strydom\footnote{\textbf{Contributions:}
  \newline \emph{The authors would like to thank no institution for
  money donated to this project. Thank you sincerely.}}}
\ead{20718284@sun.ac.za}





\address[Add1]{Stellenbosch University, Stellenbosch, South Africa}



\vspace{1cm}


\begin{keyword}
\footnotesize{
Volatility \sep ALSI \sep Standard deviation \\
\vspace{0.3cm}
}
\footnotesize{
\textit{JEL classification} L250 \sep L100
}
\end{keyword}



\vspace{0.5cm}

\end{frontmatter}



%________________________
% Header and Footers
%%%%%%%%%%%%%%%%%%%%%%%%%%%%%%%%%
\pagestyle{fancy}
\chead{}
\rhead{}
\lfoot{}
\rfoot{\footnotesize Page \thepage}
\lhead{}
%\rfoot{\footnotesize Page \thepage } % "e.g. Page 2"
\cfoot{}

%\setlength\headheight{30pt}
%%%%%%%%%%%%%%%%%%%%%%%%%%%%%%%%%
%________________________

\headsep 35pt % So that header does not go over title




\hypertarget{introduction}{%
\section{\texorpdfstring{Introduction
\label{Introduction}}{Introduction }}\label{introduction}}

In general, small cap stocks tend to have higher volatility than large
cap stocks, and mid cap stocks tend to fall in between the two in terms
of volatility. This is because small cap companies typically have less
established business models and less predictable revenue streams, which
can lead to greater fluctuations in their stock prices. On the other
hand, large cap companies tend to have more established business models
and more predictable revenue streams, which can lead to less volatility
in their stock prices. In R, one can use the package
`PerformanceAnalytics' to calculate the volatility of different indexes.

Fama \& French (\protect\hyperlink{ref-fama1997}{1997: 33}) and Grinold
\& Kahn (\protect\hyperlink{ref-grinold2000}{2000})

\hypertarget{large-cap-shares}{%
\section{\texorpdfstring{Large cap shares
\label{Large}}{Large cap shares }}\label{large-cap-shares}}

\begin{figure}[H]

{\centering \includegraphics{Volatility-of-Shares_files/figure-latex/Figure1-1} 

}

\caption{Large rolling SD \label{Figure1}}\label{fig:Figure1}
\end{figure}

\begin{figure}[H]

{\centering \includegraphics{Volatility-of-Shares_files/figure-latex/Figure2-1} 

}

\caption{Large cap drawdown chart \label{Figure2}}\label{fig:Figure2}
\end{figure}

\begin{figure}[H]

{\centering \includegraphics{Volatility-of-Shares_files/figure-latex/Figure3-1} 

}

\caption{Caption Here \label{Figure3}}\label{fig:Figure3}
\end{figure}

\hypertarget{mid-cap-shares}{%
\section{\texorpdfstring{Mid cap shares
\label{Mid}}{Mid cap shares }}\label{mid-cap-shares}}

\begin{figure}[H]

{\centering \includegraphics{Volatility-of-Shares_files/figure-latex/Figure4-1} 

}

\caption{Caption Here \label{Figure4}}\label{fig:Figure4}
\end{figure}

\begin{figure}[H]

{\centering \includegraphics{Volatility-of-Shares_files/figure-latex/Figure5-1} 

}

\caption{Caption Here \label{Figure5}}\label{fig:Figure5}
\end{figure}

\begin{figure}[H]

{\centering \includegraphics{Volatility-of-Shares_files/figure-latex/Figure6-1} 

}

\caption{Caption Here \label{Figure6}}\label{fig:Figure6}
\end{figure}

\hypertarget{small-cap-shares}{%
\section{\texorpdfstring{Small cap shares
\label{Small}}{Small cap shares }}\label{small-cap-shares}}

\begin{figure}[H]

{\centering \includegraphics{Volatility-of-Shares_files/figure-latex/Figure7-1} 

}

\caption{Caption Here \label{Figure7}}\label{fig:Figure7}
\end{figure}

\begin{figure}[H]

{\centering \includegraphics{Volatility-of-Shares_files/figure-latex/Figure8-1} 

}

\caption{Caption Here \label{Figure8}}\label{fig:Figure8}
\end{figure}

\begin{figure}[H]

{\centering \includegraphics{Volatility-of-Shares_files/figure-latex/Figure9-1} 

}

\caption{Caption Here \label{Figure9}}\label{fig:Figure9}
\end{figure}

\hypertarget{rolling-standard-deviation}{%
\section{Rolling Standard deviation}\label{rolling-standard-deviation}}

\begin{figure}[H]

{\centering \includegraphics{Volatility-of-Shares_files/figure-latex/Figure10-1} 

}

\caption{Caption Here \label{Figure10}}\label{fig:Figure10}
\end{figure}

\begin{table}[h]
\begin{center}
    \begin{tabular}{| c | c |}
    \hline
         & Standard deviation \\
        \hline
        Large cap & 0.01118614 \\
        Mid cap & 0.001418264 \\
        Small cap & 0.0002078111 \\
        \hline
    \end{tabular}
    \caption{Standard deviation}
    \label{tab:SD}
\end{center}
\end{table}

Figures \ref{Figure1} and \ref{Figure2}

\begin{table}[h]
\begin{center}
    \begin{tabular}{| c | c | c | c |}
    \hline
         & Large cap & Mid cap & Small cap \\
        \hline
        Volatility skewness & 1.067183 & 0.9504131 & 0.9286573 \\
        \hline
    \end{tabular}
    \caption{Volatility skewness}
    \label{tab:VS}
\end{center}
\end{table}

The upside/downside ratio is often used to gauge overbought and oversold
conditions in the market. Low values can indicate that the market is
reaching oversold levels, while high values can indicate that the market
is becoming overbought.

\hfill

\hypertarget{conclusion}{%
\section{Conclusion}\label{conclusion}}

I hope you find this template useful. Remember, stackoverflow is your
friend - use it to find answers to questions. Feel free to write me a
mail if you have any questions regarding the use of this package. To
cite this package, simply type citation(``Texevier'') in Rstudio to get
the citation for Katzke (\protect\hyperlink{ref-Texevier}{2017}) (Note
that uncited references in your bibtex file will not be included in
References).

\newpage

\hypertarget{references}{%
\section*{References}\label{references}}
\addcontentsline{toc}{section}{References}

\hypertarget{refs}{}
\begin{CSLReferences}{1}{0}
\leavevmode\vadjust pre{\hypertarget{ref-fama1997}{}}%
Fama, E.F. \& French, K.R. 1997. Industry costs of equity. \emph{Journal
of financial economics}. 43(2):153--193.

\leavevmode\vadjust pre{\hypertarget{ref-grinold2000}{}}%
Grinold, R.C. \& Kahn, R.N. 2000. Active portfolio management.

\leavevmode\vadjust pre{\hypertarget{ref-Texevier}{}}%
Katzke, N.F. 2017. \emph{{Texevier}: {P}ackage to create elsevier
templates for rmarkdown}. Stellenbosch, South Africa: Bureau for
Economic Research.

\end{CSLReferences}

\hypertarget{appendix}{%
\section*{Appendix}\label{appendix}}
\addcontentsline{toc}{section}{Appendix}

\hypertarget{appendix-a}{%
\subsection*{Appendix A}\label{appendix-a}}
\addcontentsline{toc}{subsection}{Appendix A}

Some appendix information here

\hypertarget{appendix-b}{%
\subsection*{Appendix B}\label{appendix-b}}
\addcontentsline{toc}{subsection}{Appendix B}

\bibliography{Tex/ref}





\end{document}
